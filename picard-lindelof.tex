\documentclass[10pt]{beamer}

\mode<presentation>
{
  \usetheme{Antibes}
  \usecolortheme{whale}
  \setbeamercovered{transparent}
}

\usepackage[english]{babel}
\usepackage{ulem}
\usepackage[utf8]{inputenc}
\usepackage{mathtools}

\renewcommand{\qed}{\hfill\blacksquare}

\title[Existence and Uniqueness Theorem of the solution to Cauchy Problem]
{Existence and Uniqueness Theorem of the solution to Cauchy Problem}

\subtitle
{Banach's fixed-point Theorem and its application}

\AtBeginSection[]
{
  \begin{frame}<beamer>{Overview}
    \tableofcontents[currentsection, currentsubsection]
  \end{frame}
}
\AtBeginSubsection[]
{
  \begin{frame}<beamer>{Overview}
    \tableofcontents[currentsection, currentsubsection]
  \end{frame}
}

\date{December 7, 2022}
\author{Prepared for Innopolis University}

\begin{document}

\begin{frame}
  \titlepage
\end{frame}

\begin{frame}{Overview}
  \tableofcontents
\end{frame}

\section{Metric spaces}
\begin{frame}
  \begin{block}{Defintion: Metric space}
    A metric space is an ordered pair \((M, d)\), in which \(M\) is a set,
    and \(d:M \times M \rightarrow \mathbb{R}\) is the metric of this set, that satisfies
    the axioms listed below.
  \end{block}

  \begin{alertblock}{Axioms}
    Let \(x, y, z \in M\), then
    \begin{itemize}
      \item \(d(x, y) \ge 0\), \(d(x, y) = 0\) iff \(x = y\);
      \item \(d(x, y) = d(y, x)\) -- symmetry;
      \item \(d(x, z) \le d(x, y) + d(y, z)\) -- triangle inequality.
    \end{itemize}
  \end{alertblock}
\end{frame}

\begin{frame}
  The Cauchy criterion is necessary and sufficient condition for a sequence \(x_n\) to converge.
  \begin{block}{Defintion: Cauchy Criterion}
    If for all \(\varepsilon > 0\) there exists a positive \(N \in \mathbb{N}\) such that
    \(d(x_n, x_m) < \varepsilon\) holds for all positive integers \(n, m > N\), then \(\{x_n\}\) converges.
  \end{block}

  \begin{block}{Defintion: Cauchy or Fundamental sequence}
    Given a metric space \((M, d)\). The sequence \(x_1, x_2, x_3, ..., x_n \in M\) is a Cauchy sequence if
    \(\{x_n\}\) satisfies Cauchy Criterion.
  \end{block}

  \begin{block}{Defintion: Complete metric space}
    A complete metric space is a metric space \((M, d)\) such that
    all Cauchy sequences consisting of elements in \(M\) also have a limit in \(M\).
  \end{block}
\end{frame}

\begin{frame}
  \begin{block}{Theorem: Completeness of \((\mathbb{R}^n, d)\)}
    The metric space \((\mathbb{R}^n, d)\) with respect to the metric \(d\)
    \[d(x, y) = \left( \sum^n_{i=0} |x_i - y_i| \right)^{1/2}\]
    for all \(x, y \in \mathbb{R}^n\) is complete.
  \end{block}
  (Hint: proof can be done using convergence of monotonic and bounded sequence and
  Bolzano-Weierstrass Theorem.)
\end{frame}

\begin{frame}
  \begin{block}{Theorem: Completeness of a closed subset}
    Let \((M, d)\) be a complete metric space, and let \(X\) be a subset of \(M\). The subspace \((X, d)\) is a complete metric space if and only if \(X\) is a closed subset of \(M\).
  \end{block}
\end{frame}

\begin{frame}
  \begin{block}{Defintion: Lipschitz continuity}
    Given two metric spaces \((X, d_X)\) and \((Y, d_Y)\), a function \(f: X \rightarrow Y\) is called
    Lipschitz continious if there exists a Lipschitz constant, \(L \ge 0\), \(L \in \mathbb{R}\), such that
    \[d_Y(f(x_1), f(x_2)) \le Ld_X(x_1, x_2)\]
    holds for all \(x_1\) and \(x_2\) in \(X\).
  \end{block}

  \begin{block}{Defintion: Contraction mapping}
    Given a metric space \((M, d)\), a Lipschitz continious function \(f: M \rightarrow M\) is called
    contraction mapping if and only if its Lipschitz constant
    \[L \in [0;1).\]
  \end{block}
\end{frame}

\begin{frame}
  \begin{block}{Theorem}
    Every contraction mapping is Lipschitz continuous and hence uniformly continuous.
  \end{block}
\end{frame}

\section{Banach's fixed-point Theorem}
\begin{frame}
  \begin{block}{Definition: Fixed-point}
    A fixed-point is a point that for the given mapping is mapped onto itself, that is
    \[f(x^*) = x^*.\]
  \end{block}

  \begin{block}{Theorem: Banach's fixed-point Theorem}
    Let an ordered pair \((M, d)\) be a non-empty complete metric space. Let \(T:M \rightarrow M\)
    be a contraction mapping onto \(M\), that is, there exists \(\alpha \in [0;1)\) such that
    \[d(T(x), T(y)) \le \alpha d(x, y)\]
    holds for all \(x, y \in M\). Then the mapping \(T\) has a unique
    fixed point \(x^* \in M\).
  \end{block}

  Also see: \url{https://www.desmos.com/calculator/vdlmrufr9w}
\end{frame}

\subsection{Proof for existence of a fixed-point}
\begin{frame}
  \textbf{Proof.} Let \((M, d)\) be a non-empty complete metric space, \(x_0\) be an arbitrary point of \(M\), and
  \(T\) be a contraction map onto \(M\).
  \vspace{0.3cm}

  \onslide<2->{Consider the sequence \(\{x_n\}\)
  \[x_0 = x_0, \, x_1 = T(x_0), \, x_2 = T(x_1), ..., x_{n + 1} = T(x_n).\]}
  \onslide<3->{Having proved that \(\{x_n\}\) is a Cauchy sequence, we will show
  that \(\{x_n\}\) is a sequence that converges in a complete metric space \(M\).
  }
  \vspace{0.3cm}

  \onslide<4->{Consider the following sequence of inequalities subject to the definition of a contraction mapping.}
  \begin{equation*}
    \begin{gathered}
      \onslide<5->{d(x_1, x_2) = d(T(x_0), T(x_1)) \le \alpha d(x_0, x_1),   \\}
      \onslide<6->{d(x_2, x_3) = d(T(x_1), T(x_2)) \le \alpha^2 d(x_0, x_1), \\}
      \onslide<7->{d(x_3, x_4) = d(T(x_2), T(x_3)) \le \alpha^3 d(x_0, x_1), \\
      \vdots}
    \end{gathered}
  \end{equation*}
\end{frame}

\begin{frame}
  \textbf{Assumption.} \(d(x_n, x_{n + 1}) \le \alpha^n d(x_0, x_1)\). We will prove this assumption
  by mathematical induction.
  \vspace{0.3cm}

  \onslide<2->{We have established that \(d(x_1, x_2) \le \alpha d(x_0, x_1)\) is true for \(n = 1\).}
  \onslide<3->{Suppose \[d(x_k, x_{k + 1}) \le \alpha^k d(x_0, x_1)\] is true for \(k\).}
  \onslide<4->{Let us show that our assumption is true for \(k + 1\).}
  \onslide<5->{\[d(x_{k + 1}, x_{k + 2}) = d(T(x_k), T(x_{k + 1})) \le \alpha \uwave{d(x_k, x_{k + 1})} \le
  \alpha^{k + 1} d(x_0, x_1).\]}
  \onslide<6->{Therefore, \(d(x_n, x_{n + 1}) \le \alpha^n d(x_0, x_1)\)
  true for any \(n \in \mathbb{N}\).}
\end{frame}

\begin{frame}
  \textbf{Proof (cont).} Apply the triangle inequality to \(d(x_n, x_m)\), \(n, m \in \mathbb{N}\), and \(m > n\).
  \begin{align*}
    \onslide<2->{d(x_n, x_m) &\le d(x_n, x_{n+1}) + d(x_{n+1}, x_{n+2}) + ... + d(x_{m - 1} , x_m) \le \\}
    \onslide<3->{&\le \alpha^n d(x_0, x_1) + \alpha^{n+1} d(x_0, x_1) + ... + \alpha^{m - 1} d(x_0, x_1) = \\}
    \onslide<4->{&= \sum_{k = 0}^{m - n - 1} \alpha^k \cdot \alpha^{n}d(x_0, x_1) \le \\}
    \onslide<5->{&\le \sum_{k = 0}^{\infty} \alpha^k \cdot \alpha^{n}d(x_0, x_1)
    =^* \frac{1}{1 - \alpha}\alpha^{n}d(x_0, x_1),}
  \end{align*}
  \onslide<6->{*\(|\alpha| < 1\) by the definition of contraction mapping.}
\end{frame}

\begin{frame}
  \textbf{Proof (cont).} Let \(m > n > N\), \(N \in \mathbb{N}\). Having
  \[d(x_n, x_m) \le \frac{1}{1 - \alpha}\alpha^{n}d(x_0, x_1),\]
  we tend to \(N \rightarrow \infty\), then \(n \rightarrow \infty\) and \(m \rightarrow \infty\).
  \onslide<2->{\[\lim_{n \rightarrow \infty} \frac{\alpha^{n}}{1 - \alpha}d(x_0, x_1) = 0,\]
  since \(|\alpha| < 1\).} \onslide<4->{Therefore, we get that for \(N \rightarrow \infty\)
  \[0 \le d(x_n, x_m) \le 0 \Rightarrow d(x_n, x_m) = 0.\]}
  \onslide<4->{This means that for an arbitrary \(\varepsilon > 0\) there exists \(N_0 \in \mathbb{N}\) such that
  for \(N > N_0\) the following holds}
  \onslide<5->{\[\frac{\alpha^{n}}{1 - \alpha}d(x_0, x_1) < \varepsilon} \onslide<7->{\xRightarrow[]{d(x_0, x_1) \ne 0}
  \frac{\alpha^{n}}{1 - \alpha} < \frac{\varepsilon}{d(x_0, x_1)}.\]}
\end{frame}

\begin{frame}
  \textbf{Proof (cont).} Next, we get that
  \[d(x_n, x_m) \le \frac{\alpha^{n}}{1 - \alpha}d(x_0, x_1) \le
  \frac{\varepsilon}{d(x_0, x_1)} d(x_0, x_1) = \varepsilon.\]
  \onslide<2->{Therefore, since \(m, n, N\) are arbitrary, \(\{x_n\}\) is a Cauchy sequence.}
  \vspace{0.3cm}

  \onslide<3->{Since \(\{x_n\}\) is an arbitrary Cauchy sequence in the complete metric space \(M\),
  for \(n \rightarrow \infty\), \(x_n \rightarrow x^* \in M\).}
  \vspace{0.3cm}

  \onslide<4->{Now we will prove that limit \(x^*\) is a fixed point.}
\end{frame}

\begin{frame}
  \textbf{Proof (cont).} We have that \(x_{n + 1} = T(x_{n})\). As we tend \(n \rightarrow \infty\), we get
  \begin{align*}
    \onslide<2->{\lim_{n \rightarrow \infty}x_{n + 1} &= \lim_{n \rightarrow \infty}T(x_{n}), \\}
    \onslide<3->{x^* &= \lim_{n \rightarrow \infty}T(x_{n}) = T(\lim_{n \rightarrow \infty}x_{n}) = T(x^*)}
  \end{align*}
  \onslide<3->{since \(T\) is a continuous function.}
  \vspace{0.3cm}

  \onslide<4->{This proves that in a non-empty complete metric space \((M, d)\)
  with the contraction map \(T\) there exists a fixed point \(x^* = T(x^*)\). \\ \(\qed\)}
\end{frame}

\subsection{Proof for uniqueness of a fixed-point}
\begin{frame}
  \textbf{Proof.} We will now prove the uniqueness of the fixed point \(x^*\).
  Using proof by contradiction, suppose that there is another fixed point \(z^*\)
  different from \(x^*\). \onslide<2->{Then let us find \(d(x^*, z^*)\).
  \[d(x^*, z^*) = d(T(x^*), T(z^*)) \le \alpha d(x^*, z^*),\]}
  \onslide<3->{and since \(|\alpha| < 1\)}
  \begin{gather*}
    \onslide<4->{d(x^*, z^*) \le \alpha d(x^*, z^*) < d(x^*, z^*), \\}
    \onslide<5->{d(x^*, z^*) < d(x^*, z^*).}
  \end{gather*}
  \onslide<6->{This is a contradiction, and hence the assumption that
  there is another different fixed point is not true.}
  \onslide<7->{Therefore, the fixed point \(x^*\) is unique. \\ \(\qed\)}
\end{frame}

\section{Pointwise and Uniform convergence}
\begin{frame}
  \begin{block}{Theorem: Pointwise convergence}
    To say that \(f_n(x) \rightarrow f(x)\) pointwise is to say that
    \[|f_n(x) - f(x)| \rightarrow 0 \text{, as } n \rightarrow \infty.\]
  \end{block}

  \begin{block}{Theorem: Uniform convergence}
    To say that \(f_n(x) \rightarrow f(x)\) uniformly on \(X\) is to say that
    \[\sup_{x \in X}|f_n(x) - f(x)| \rightarrow 0 \text{, as } n \rightarrow \infty.\]
  \end{block}
\end{frame}

\begin{frame}
  \begin{block}{Theorem: Uniform convergence of the continious function}
    If the sequence of continious functions \(f_n(x) \rightarrow f(x)\) uniformly on \(X\),
    then \(f(x)\) is continious on \(X\).
  \end{block}

  \textbf{Proof.} Let \(\varepsilon > 0\). Then there exists an integer \(N\) such that
  for all integers \(n \ge N\)
  \[\sup_{x \in X}|f_n(x) - f(x)| < \frac{\varepsilon}{3}.\]
  Let \(\delta > 0\). Suppose that \(\|x - y\| < \delta\) for all \(x, y \in X\), then because of the continuity of \(f_n\)
  \[\|f_N(x) - f_N(y)\| < \frac{\varepsilon}{3}.\]
  Finally, using triangle inequality, we have for all \(x, y \in X\)
  \[\|f(x) - f(y)\| \le \|f(x) - f_N(x)\| + \|f_N(x) - f_N(y)\| + \|f_N(y) - f(y)\| < \varepsilon.\]
  Consequently, \(f\) is continuos at every point \(x \in X\). \\ \(\qed\)
\end{frame}

\section{Normed spaces}
\begin{frame}
  \begin{block}{Definition: Normed space}
    A normed space is an ordered pair \((V, \| \cdot \|)\), where \(V\) is a vector space,
    and \(\| \cdot \|:V \rightarrow \mathbb{R}\) is the norm of this space, that satisfies
    the axioms listed below.
  \end{block}

  \begin{alertblock}{Axioms}
    Let \(x, y \in V\), then
    \begin{itemize}
      \item \(\|x\| \ge 0\), \(\|x\| = 0\) implies \(x = 0\);
      \item \(\| \lambda x \| = |\lambda| \cdot \|x\|\), for all scalars \(\lambda\);
      \item \(\|x + y\| \ge \|x\| + \|y\|\) -- triangle inequality.
    \end{itemize}
  \end{alertblock}

  \begin{exampleblock}{Corollary}
    Any normed space \((V, \| \cdot \|)\) is a metric space with the metric
    \(d(x, y) = \| x - y \|\) on \(V\).
  \end{exampleblock}
\end{frame}

\section{Space of continuous functions}
\begin{frame}
  \begin{block}{Space of continuous functions}
    A space of continuous functions \(C[a;b]\) is a normed space whose elements are
    functions continuous on the interval \([a;b]\). The norm of such a space for \(y \in C[a;b]\) is defined as
    uniform norm
    \[\|y\| = \sup_{x \in [a;b]}|y(x)|.\]
  \end{block}
  Using the norm in this form, we can define a limit for uniform convergence of \(\{f_n\}\) on \(X\)
  \[\lim_{n \rightarrow \infty} \|f_n - f\| \rightarrow 0.\]
\end{frame}

\begin{frame}
  \begin{block}{Theorem: Completeness of \(C[a,b]\) with respect to the uniform norm}
    A metric space \((C[a;b], d)\) of continuous functions on an interval \([a; b]\)
    with the uniform metric
    \[d(f, g) = \sup_{x \in [a,b]}|f(x) - g(x)|\]
    is complete.
  \end{block}

  \textbf{Proof.} Consider an arbitrary fundamental sequence \(\{f_n\}\), \(n \in \mathbb{N}\), from \(C[a;b]\).
  Let \(\varepsilon > 0\), then there exists \(N \in \mathbb{N}\) such that for all integers \(n, m > N\)
  \[\sup_{x \in [a,b]}|f_n(x) - f_m(x)| < \frac{\varepsilon}{2}.\]
  This implies that \(\{f_n(x)\}\) is a Cauchy sequence in \(\mathbb{R}\) for each fixed \(x \in [a, b]\).
  Since \(\mathbb{R}\) is complete, \(\{f_n(x)\}\) has a limit, which is also in \(\mathbb{R}\).
  \vspace{0.3cm}

  We will call this limit \(f(x)\).
\end{frame}

\begin{frame}
  \textbf{Proof (cont).} Then \(f_n(x) \rightarrow f(x)\) for each fixed \(x \in [a;b]\), that is
  for all \(\varepsilon > 0\), there exists a positive \(M \in \mathbb{N}\) such that for all natural \(m > M\)
  \[|f_m(x) - f(x)| < \frac{\varepsilon}{2}.\]

  Using triangle inequality, for all \(x \in X\)
  \[|f_n(x) - f(x)| \le |f_n(x) - f_m(x)| + |f_m(x) - f(x)| < \varepsilon.\]
  It follows that for \(n \rightarrow \infty\), \(d(f_n(x), f(x)) \rightarrow 0\).
  Therefore, \(f_n \rightarrow f\) uniformly on the segment \([a;b]\). Uniform convergence of
  continious function implies that \(f\) is also continious on \([a;b]\).
  \vspace{0.3cm}

  We have shown that an arbitrary Cauchy sequence in the metric space of continuous functions \((C[a, b], d)\)
  on the interval \([a, b]\) with metric \(d\) has a limit also in \(C[a, b]\). Therefore, \((C[a, b], d)\) is
  complete metric space. \\ \(\qed\)
\end{frame}

\section{Picard–Lindelöf Theorem}
\begin{frame}
  \begin{block}{Theorem: Picard–Lindelöf Theorem}
    Let \(f(x,y)\) be a function continuous on a rectangle
    \[R=\{(x,y):|x_0 - x| \le a, |y_0 - y| \le b\},\]
    and hence, bounded in \(R\), \(|f(x,y)| \le M\).
    Suppose that \(f\) is Lipschitz continuous on \(R\) with respect to its second
    argument \(y\), meaning that
    \[|f(x, y) - f(x, z)| \le L|y - z|\]
    for all \((x, y), \, (x, z) \in R\).
    Then, there exists positive \(\varepsilon = \min{\{a, \frac{b}{M}, \frac{1}{L}\}}\) such that
    Cauchy problem
    \[\begin{cases}y' = f(x,y) \\ y(x_0) = y_0 \end{cases}\]
    has a unique solution solution on the interval
    \([x_0 - \varepsilon; x_0 + \varepsilon]\).
  \end{block}
\end{frame}

\begin{frame}
  We will divide the proof of the Picard-Lindelöf Theorem into several parts:
  \begin{itemize}
    \item transform the Cauchy problem into an equivalent integral equation;
    \item ...show the existence of a contraction mapping;
    \item ...apply the Banach fixed point Theorem.
  \end{itemize}
\end{frame}

\begin{frame}
  \textbf{Proof. Part 1.} If \(y(x) \in C[x_0 - a;x_0 + a]\) is a solution of the given
  Cauchy problem, then by integrating \(y' = f(x, y)\) we get
  \onslide<2->{\[\int_{x_0}^{x} y'(t) dt = \int_{x_0}^{x} f(t,y(t)) dt,\]}
  \onslide<3->{then from the Fundamental Theorem of Calculus follows
  \[y(x) = y_0 + \int_{x_0}^{x} f(t,y(t)) dt.\]}
  \onslide<4->{Let us denote the right hand side of the latter equation as \(\mathcal{T}\), that is}
  \onslide<5->{\[\mathcal{T}(y(x)) := y_0 + \int_{x_0}^{x} f(t,y(t)) dt, \, x \in \mathcal{I},\]}
  \onslide<6->{where \(\mathcal{I} := [x_0 - \varepsilon; x_0 + \varepsilon]\) is an interval for which
  solutions to the given Cauchy problem are defined within the rectangle \(R\) for every \(x \in \mathcal{I}\).}
\end{frame}

\begin{frame}
  \textbf{Remark.} Notice that
  \[|f(x, y)| \le M \Leftrightarrow |y'| \le M.\]
  This is means that tangent of tangent slope does not exceed \(M\). Since we analyze function
  continious on the interval \(\mathcal{I}\) within the rectangle \(R\) with sides \(2a\) and \(2b\)
  \[|y'| \le M \le \frac{2b}{2\varepsilon} = \frac{b}{\varepsilon}.\]
  From this follow an additional upper bound for \(\varepsilon\).
  \[\varepsilon \le \frac{b}{M}.\]
\end{frame}

\begin{frame}
  \textbf{Motivation for the further steps.} As every solution to the given Cauchy problem
  corresponds to a fixed point of \(\mathcal{T}\),
  showing that \(\mathcal{T}\) is a contraction mapping of the complete
  metric space \((\mathcal{X}, d)\) is sufficient to prove existence and uniqueness
  of this solution by Banach’s fixed-point Theorem.
  \vspace{0.3cm}

  \onslide<2->{We will build the set \(\mathcal{X}\) is such a way, that it contains solutions to the given Cauchy problem
  that are defined within the rectangle \(R\) for every \(x \in \mathcal{I}\).}
\end{frame}

\begin{frame}
  \textbf{Proof (cont).} We will build \(\mathcal{X}\) as a subspace of \(C(\mathcal{I})\).
  Let us bound all solutions \(y(x)\), so that they do not go outside the rectangle \(R\) at all points of the
  interval \([x_0 - \varepsilon; x_0 + \varepsilon]\).
  \onslide<2->{\[|y_0 - y(x)| = \left|\int_{x_0}^{x} f(t,y(t)) dt\right| \le |x_0 - x| \cdot |\max_{x \in \mathcal{I}}{f(x,y(x))}|
  \le \varepsilon M.\]}
  \onslide<3->{Then
  \[\mathcal{X} := \{y \in C(\mathcal{I}) : y(x_0) = y_0, \, \sup_{x \in \mathcal{I}}|y_0 - y(x)| \le \varepsilon M\}.\]}

  \onslide<4->{Since \(\mathcal{X}\) is a closed subspace of \(C\),
  metric space \((\mathcal{X}, d)\) is complete with respect to the uniform metric \(d\).}
\end{frame}

\begin{frame}
  \textbf{Proof. Part 2.} To show that \(\mathcal{T}\) is a contraction mapping is to show that
  \[\mathcal{T}: \mathcal{X} \rightarrow \mathcal{X}\] and that with \(\alpha \in [0;1)\)
  \[d(\mathcal{T}(g), \mathcal{T}(h)) \le \alpha d(g, h)\]
  holds for all \(g, h \in \mathcal{X}\).
\end{frame}

\begin{frame}
  \textbf{Proof (cont).} We claim that \(\mathcal{T}: \mathcal{X} \rightarrow \mathcal{X}\).
  Firstly, consider arbitrary \(g \in \mathcal{X}\), then we need to show \(\mathcal{T}\) that
  does not strive away from the point \((x_0, y_0)\) ouside the rectangle \(R\)
  \onslide<2->{\[|y_0 - \mathcal{T}(g(x))| = \left|\int_{x_0}^{x} f(t,g(t)) dt\right| \le |x_0 - x|
  \cdot |\max_{x \in \mathcal{I}}{f(x,g(x))}| \le \varepsilon M.\]}
  \onslide<3->{So, \(\mathcal{T}(g(x))\) is a function passing through \((x_0, y_0)\) and
  continious within \(R\) on \(\mathcal{I}\).}
  \onslide<4->{Therefore, we showed that \(\mathcal{T}: \mathcal{X} \rightarrow \mathcal{X}\), indeed.}
\end{frame}

\begin{frame}
  \textbf{Proof (cont).} Now, we we claim that \(\mathcal{T}\) is a contraction mapping.
  Consider arbitrary \(g, h \in \mathcal{X}\). Then,
  \onslide<2->{\[d(\mathcal{T}(g), \mathcal{T}(h)) = |\mathcal{T}(g) - \mathcal{T}(h)| \le\]}
  \onslide<3->{\[\le \int_{x_0}^x f(t, g(t)) - g(t, h(t)) dt \le\]}
  \onslide<4->{\[\le |x_0 - x| \max_{x \in \mathcal{I}}{|f(x, g(x)) - g(x, h(x))|} \le\]}
  \onslide<5->{\[\le |x_0 - x| \max_{x \in \mathcal{I}}{L|g(x) - h(x)|} \le\]}
  \onslide<6->{\[\le \varepsilon L d(g, h).\]}

  \onslide<7->{Since we wish to show that \(\mathcal{T}\) is a contraction mapping, \(0 \le \varepsilon L < 1\).}
  \onslide<8->{Therefore, \(\mathcal{T}\) is a contraction mapping on \(\mathcal{X}\) when \(\varepsilon < \frac{1}{L}\).}
\end{frame}

\begin{frame}
  \textbf{Proof. Part 3.} We have shown that complete metric space \((\mathcal{X}, d)\)
  has a contraction mapping \(\mathcal{T}\) on the interval \([x_0 - \varepsilon;x_0 + \varepsilon]\) on \(\mathcal{X}\),
  where \(\varepsilon = \min{\{a, \frac{b}{M}, \frac{1}{L}\}}\).
  Banach's fixed-point Theorem then implies that \(\mathcal{T}\) has a unique
  fixed point \(y(x) \in \mathcal{X}\) such that
  \[y(x) = y_0 + \int_{x_0}^{x} f(t,y(t)) dt.\]

  \onslide<2->{\textbf{Conclusion:} It thus follows that the given initial value problem has a unique continuous solution
  \(y(x)\) on the interval \([x_0 - \varepsilon;x_0 + \varepsilon]\) on \(\mathcal{X}\),
  where \(\varepsilon = \min{\{a, \frac{b}{M}, \frac{1}{L}\}}\). \\ \(\qed\)}
\end{frame}

\section{References}
\begin{frame}
  \nocite{*}
  \frametitle{References}
  \bibliographystyle{apalike}
  \bibliography{references}
\end{frame}

\end{document}
